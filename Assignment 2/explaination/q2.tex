\section{Search Engine}

\subsection{Construct Document Graph}

\begin{figure}[htbp]
    \centering
    \includegraphics[width=.8\textwidth]{output/cosine-similarity-histogram.png}
    
    \caption{Histogram of Cosine Similarities}
    \label{fig:cosine-similarities}
\end{figure}

Cosine similarity measures the angle between two vectors (i.e. TF-IDF vectors of SMS messages). 
It ranges from 0.0 (where messages with no shared words) to 1.0 (where messages are identical in terms of content).

Figure \ref{fig:cosine-similarities} plots the frequency of cosine similarity value between SMS messages. 
Here we see a histogram that is right skewed towards low similarities (0.0 - 0.2), which suggests that most messages are unrelated, and that the dataset is diverse with many topics. 

The similarlity threshold can be set on graph construction which affects which nodes (messages) are connected.
In Figure \ref{fig:document-similarity-graph}, the outer ring shows the nodes of messages that have cosine similarity below the threshold and the core shows the connected nodes with edges.
A high threshold (e.g. 0.8) will only connect highly similar messages, shown in Figure \ref{fig:document-similarity-graph-0.8}, where a low threshold (e.g. 0.2), more messages will be linked and creating a densely connected graph, shown in Figure \ref{fig:document-similarity-graph-0.2}.


\begin{figure}[htbp]
    \centering
    \begin{subfigure}[b]{.8\textwidth}
        \centering
        \includegraphics[width=.9\textwidth]{output/document-similarity-threshold-0.2.png} % replace with your image file
        \caption{Threshold: 0.2}
        \label{fig:document-similarity-graph-0.2}
    \end{subfigure}
    \hfill
    \begin{subfigure}[b]{.8\textwidth}
        \centering
        \includegraphics[width=.9\textwidth]{output/document-similarity-threshold-0.8.png} % replace with your image file
        \caption{Threshold: 0.8}
        \label{fig:document-similarity-graph-0.8}
    \end{subfigure}
    
    \caption{Document Similarity Graph (Spam vs Ham)}
    \label{fig:document-similarity-graph}
\end{figure}



\subsection{PageRank Algorithm}

PageRank is used to rank pages (nodes) based on their importance. 
As the edges are connected to the message nodes based on cosine similarity, only with similarity greater than threshold. 
This means that a message node becomes important if many important or similar nodes point to it.

\begin{table}[h]
    \centering
    \pgfplotstabletypeset[
        col sep=tab,
        columns/PageRank/.style={fixed,precision=20},
        columns/Label/.style={string type},
        columns/Message/.style={string type},
        every head row/.style={before row=\hline, after row=\hline},
        every last row/.style={after row=\hline},
        column type={lll},
    ]{output/ranked_sms_messages.csv}
    \caption{Top 20 Messages based on PageRank with threshold of 0.8}
    \label{top-page-rank-table} % Label for referencing
\end{table}

In Table \ref{top-page-rank-table}, these messages have high PageRank score, which means that they are well-connected in terms of context. 
The top 20 messages are labelled `Ham', which further contribute to contextually relavent content.
They have a common topics of casual conversation and personal updates and are usually of shorter length and straight to the point in conversations.


\begin{landscape}
\begin{table}[h]
    \centering
    \pgfplotstabletypeset[
        col sep=tab,
        columns/PageRank/.style={fixed,precision=20},
        columns/Label/.style={string type},
        columns/Message/.style={string type},
        every head row/.style={before row=\hline, after row=\hline},
        every last row/.style={after row=\hline},
        column type={lll},
    ]{output/bottom_ranked_sms_messages.csv}
    \caption{Bottom 20 Messages based on PageRank with threshold of 0.8 (Descending order)}
    \label{bottom-page-rank-table} % Label for referencing
\end{table}

In Table \ref{bottom-page-rank-table}, these messages have low PageRank score, which means that they are less important.
The bottom 20 messages have some messages labelled as `Spam', which are typically standalone and contextually irrelavent, and therefore ranks lower and do not contribute to the network relavence. 
They have a common topics of promotions, unsolicited offers, mobile phone scams and are often longer in length with many keywords to capture attention: URGENT, WON, CALL, etc.
\end{landscape}


