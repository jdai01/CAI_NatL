\section{Zipf's Law}

\subsection*{List of unique words sorted to their frequency in descending order}
Shown in accompanying Jupyter Notebook.


\subsection*{Discussion on findings}
Zipf's Law states that in a large collection of words, the frequency of any word is inversely proportional to its rank:
\begin{align}
    f \propto \frac{1}{r}
\end{align}
where  
\begin{itemize}
    \item $f$: frequency of word  
    \item $r$: rank of the word
\end{itemize}

This means that the most frequent word occurs twice as often as the second most frequent, three times as often as the third most frequent, and so on.

With reference to the jungle book dataset, words that are short in length occur in higher frequencies compared to words with longer lengths. 
For example, ``the'', ``and'', ``of'' occur in higher frequency compared to ``produce'', ``subscribe'', ``newsletter'' of lower frequency.
\newline

To verify Zipf's law on a textual corpus, using the jungle book dataset, chi-square goodness of fit test is performed \cite{web:chi_sq_test}.  
\begin{itemize}
    \item $H_0$: The observed frequencies are equal to the expected frequencies.  
    \item $H_1$: The observed frequencies are not equal to the expected frequencies. 
\end{itemize}

\begin{align*}
    \chi^2_{\text{statistics}} & = 19760.11496822835 \\
    \chi^2_{\text{critical}} & = 5107.674219300448 \\
    \chi^2_{\text{statistics}} & > \chi^2_{\text{critical}} & (\text{Reject } H_0)
\end{align*}

As the $H_0$ is rejected, this suggests that observed frequencies are not equal to the expected frequencies, and that Zipf's Law does not hold.


\begin{figure}[h!]
    \centering
    \includegraphics[width=\textwidth]{output/q1_plot.png}
    \caption{
        Linear and Log-Log Curves.\\
        Observed line (blue) follows the frequency of words observed in the jungle book dataset.
        Expected line (red) follows the theoretical word frequency according to Zipf's law.
    }
    \label{fig:linear_log_curve}
\end{figure}

With reference to Figure~\ref{fig:linear_log_curve}, the linear curve is not a good presentation of the Zipf's law as the frequency of occurance is happening at an exponentially decreasing rate.
However, there is a general downward trend of frequency compared to rank in the Log-Log Curve.
The statistical analysis shows that the observed frequency is not equal to the expected frequency, suggests that the jungle book dataset contains words that does not follow Zipf's Law significantly.
If Zipf's Law holds, the observed line (blue) should form a straight line with a slope close to -1, as depicted by the expected line (red) in Log-Log curve. 
However, the observed line does not follow a slope close to -1 in Log-Log curve. 
